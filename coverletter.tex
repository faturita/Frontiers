Dear Editor, 

We are submitting this Original Article, titled "Histogram of Gradient Orientations of Signal Plots applied to P300 Detection" which we believe is within the scope of the Research Topic "Computational Methodologies in Brain Imaging and Connectivity: EEG and MEG Applications".  This article encompass part of the work that we have been doing in the context of the setup of a Brain Computer Interfaces Laboratory at the Computer Engineering Department of the ITBA University in Buenos Aires Argentina, and is part of a PhD Thesis of the first author.  This is original work, which has not been submitted for publication elsewhere.

In this work we are proposing a novel method to analyze EEG signals, particularly the Event Related Potentials P300, but based on the study of signal waveforms extracted from signal plots.  We developed this approach while working on a Brain Computer Interface assistive device and quickly realized that the idea had more ramifications into clinical EEG where there is a vast body of knowledge and experience which is based on visual inspection of EEG signal waveforms.

We verify our approach by offline processing a public dataset of ALS patients.  Additionally, we replicated the same experiment conditions on healthy volunteers and obtained a new dataset of healthy subjects that we used for comparison.  

We believe that this proposed method based on the histograms of gradients of signal plots, and most importantly, its  application to identify ERP markers within EEG signals is relevant for this special issue and that it will contribute with a novel tool that can help as an alternative mechanism to study the brain.

Yours truly
Rodrigo Ramele
PhD Candidate
ITBA University 
Argentina